\chapter{Introdução}

\section{Motivação}

Gerenciar adequadamente os dados armazenados em um banco de dados é fundamental para qualquer sistema de informação. Tal gerenciamento precisa eficiente e intuitivo, de maneira a evitar alterações equivocadas e garantir que os usuários do sistema interajam com dados atualizados e corretos.

Com esse objetivo foram desenvolvidos diferentes interfaces para sistemas de gerenciamento de bancos de dados, com formulários relacionados {a} estrutura de tabelas definidos no schema do banco, e possibilidade de interagir com os dados através de uma linguagem de DCL (Data Control Language / Linguagem de Controle de Dados). Soluções como phpMyAdmin, SQL Server Management Studio, ou o Oracle RDBMS trazem essas funcionalidades, porém exclusivamente para bancos de dados SQL.

Para bancos de dados em grafo, no entanto, em especial o Neo4j Database, as ferramentas de gerenciamento de dados disponíveis se limitam à interações diretas por DCL (Neo4j Browser). Sendo assim, geram abertura para alterações equivocadas nos dados e não permitem uma fácil visualização e gerenciamento intuitivo dos dados, especialmente quando utilizados por usuários ingênuos ao schema.

\section{Contribuição}

O presente TCC apresenta uma interface de gerenciamento de dados para bancos de dados Neo4j, através de uma API em Graphql, com foco na visualização, navegação e gerenciamento intuitivo dos dados do grafo armazenado. A interface recebe o schema do banco de dados e gera um ambiente de "Perfis e Listas", que permite a criação e edição de nós, seus relacionamentos, e os dados de cada um destes. Hoje o sistema é usado amplamente por mais de 30 colaboradores principalmente dos times de Qualidade, Desenvolvimento, Suporte e Educacional da empresa para qual foi desenvolvido.

O trabalho documenta o funcionamento e a arquitetura do sistema, tanto a interface do usuário, que é gerada através de componentes reutilizáveis e as ferramentas Nuxt.js, Vue.js e Apollo Client, como a camada intermediária que conecta a interface ao banco de dados. Para tal, utiliza-se o o Node.js, Express.js, e a biblioteca oficial do Neo4j que conecta o banco de dados à uma API em GraphQL, @neo4j/graphql.

A interface do usuário segue uma dinâmica de navegação diferente, onde é mapeado o isomorfismo de um nó e seus vizinhos com uma página de perfil, suas abas e seus elementos, de maneira que conseguimos gerar uma página de perfil para cada um dos nós no banco e assim navegar facilmente através de suas relações para os perfis de outros nós - alterando ou acrescentando dados arbitrariamente no processo.

\section{Organização}

Este TCC está estruturado da seguinte forma:
\begin{itemize}
  \item Soluções existentes de gerenciamento de dados para bancos de dados Neo4j, explicitando a falta de uma solução como a proposta no trabalho.
  \item Um resumo das principais ferramentas e tecnologias utilizadas no desenvolvimento do sistema proposto.
  \item Uma descrição do funcionamento das partes do sistema.
  \item Como o sistema é utilizado na empresa Jovens Gênios
\end{itemize}

OBS: O domínio dos dados da Jovens Gênios consiste principalmente das estruturas escolares (Redes, Escolas, Turmas), as pessoas envolvidas no processo de ensino-aprendizagem (Professores, Gestores, Alunos), além de conteúdos didáticos, organizados numa hierarquia em árvore, uma para cada disciplina definida na BNCC (Base Nacional Comum Curricular).
