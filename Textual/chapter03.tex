\chapter{Tecnologias Utilizadas}
\label{chap3}

\section{Neo4j}
O Neo4j é um sistema de gerenciamento de banco de dados orientado a grafos desenvolvido pela Neo4j Inc., que opera sob uma estrutura de dados que consiste em nós, relacionamentos e propriedades. Ao contrário dos bancos de dados relacionais, que usam tabelas e linhas, o Neo4j permite que as informações sejam armazenadas em um formato altamente conectado, imitando as interações do mundo real. Isso faz do Neo4j uma escolha ideal para cenários onde as relações entre os dados são tão importantes quanto os próprios dados.

 
Cada Nó e cada Aresta possui um ou mais Rótulos (\textit{labels}), que definem o \textit{tipo} do dado e instanciam um index de lookup, funcionando similar à uma tabela num banco SQL. Podemos eficientemente recuperar os dados de todos os nós ou todas as arestas de um certo rótulo para listá-las, por exemplo.

Cada rótulo possui sua definição de \textit{tipo}, que define a tipagem de cada uma de suas propriedades, incluindo possíveis ligações com nós de mesmo, ou outro, rótulo.

\subsection{Cypher Query Language}

O banco Neo4j, diferentemente da maioria dos bancos relacionais, não utiliza SQL como a linguagem de manipulação de registros de dados, e sim a própria linguagem chamada Cypher.

\begin{figure}[H]
    \centering
    \includegraphics[width=1.0\linewidth]{Imagens/chap03/cypher-exemple.png}
    \caption{Exemplo Cypher https://neo4j.com/developer/cypher/.}
    \label{fig:profile-exemple}
\end{figure}
https://neo4j.com/developer/cypher/

\section{GraphQL}


\subsection{@neo4j/graphql}
A GraphQL to Cypher query execution layer for Neo4j and JavaScript GraphQL implementations.

Para realizar a conexão entre as requisições em GraphQL que a interface do usuário irá realizar com o banco de dados, precisamos realizar essa tradução para a Cypher Query Language que será executada no banco. Para tal trabalho a neo4j disponibiliza uma biblioteca que permite tanto definir e gerar o schema do banco de dados, como gerar automaticamente requisições de CRUD para cada um dos \textit{tipos} definidos.

Cada rótulo possui sua definição de \textit{tipo}, que define a tipagem de cada uma de suas propriedades, incluindo possíveis ligações com nós de mesmo, ou outro, rótulo.

\section{Apollo}

\section{Node.js}

\subsection{Express.js}

\section{Vue.js}
