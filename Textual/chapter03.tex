\chapter{Tecnologias Utilizadas}
\label{chap3}

\section{Neo4j}
O Neo4j é um banco de dados NoSQL orientado a grafos, que opera sob uma estrutura de dados que consiste em nós, relacionamentos e propriedades. Ao contrário dos bancos de dados relacionais, que usam tabelas e linhas, o Neo4j permite que as informações sejam armazenadas em um formato altamente conectado, imitando as interações do mundo real. Isso faz do Neo4j uma escolha ideal para cenários onde as relações entre os dados são tão importantes quanto os próprios dados.

O Neo4j é um sistema de gerenciamento de banco de dados orientado a grafos desenvolvido pela Neo4j Inc. Seus elementos de dados consiste em nós, relacionamentos e propriedades. Ao contrário dos bancos de dados relacionais, que usam tabelas e linhas
Cada Nó e cada Aresta possui um ou mais “Label”s (Rótulos), que definem o tipo do dado e instanciam um index de lookup, funcionando similar à uma tabela num banco SQL.

Cada label possui sua definição com suas propriedades, incluindo possíveis arestas e ligações com nós de outras labels. Tais definições de tipos são determinada seguindo a sintaxe da biblioteca @neo4j/graphql, onde escrevemos e documentamo-os, e são utilizados para gerar o Schema do Banco de Dados.

O caso de uso da Jovens Gênios inclui nós das estruturas escolares (Redes de ensino, Escolas, Turmas, Alunos, Professores, Gestores), nós do conteúdo gerado pela empresa (Questões, Resumos, Tópicos, Cursos, Disciplinas), nós das atividades que acontecem nas plataformas (Desafio, Tarefa, Prova Somativa, Prova Diagnóstica, Campeonato, Aula Invertida), e, principalmente, os nós referentes às respostas dos alunos (StudentAnswer), que conectam o aluno à questão e ao contexto da atividade que levou o aluno àquela questão em primeiro lugar.

A estrutura em árvore dos tópicos e da estrutura escolar, assim como a facilidade de realizar a manutenção e desenvolver novas funcionalidades utilizando a biblioteca @neo4j/graphql, foram os fatores determinantes para a escolha desse banco.
\subsection{Cypher Query Language}

O banco Neo4j, diferentemente da maioria dos bancos relacionais, não utiliza SQL como a linguagem de manipulação de registros de dados, e sim a própria linguagem chamada Cypher.

![https://neo4j.com/developer/cypher/](https://s3-us-west-2.amazonaws.com/secure.notion-static.com/2d349383-c171-4b7a-954f-6aef29f1d5da/Untitled.png)

https://neo4j.com/developer/cypher/

O objetivo do frontend é gerar uma interface que o usuário ingênuo consegue interagir, e para cada ação realizada um Cypher é criado que realiza as manipulações requeridas de maneira segura e eficiente. Tal Cypher então é executado no nosso banco de dados.

Não é o frontend que gera tal cypher, e sim o nosso backend em Node.js que recebe as requisições do front, realiza essa tradução da requisição para um Cypher, e executa no banco de dados atraves do Driver e de métodos seguros de autenticação e regras de acesso.

A requisição que é enviada é no padrão GraphQL, e a biblioteca @neo4j/graphql facilita nosso trabalho automaticamente (dado as definições de tipos estabelecidas) gerando o cypher resultante dado a requisição.
\section{GraphQL}

Lorem ipsum dolor sit amet, consectetur adipiscing elit. Nunc et rutrum tortor. Aenean placerat sed erat at posuere. Praesent a dui augue. Etiam ultrices est in eleifend convallis. Nulla condimentum eleifend nunc, quis commodo nisi imperdiet a. Vestibulum dolor neque, rutrum ac cursus vitae, facilisis et felis.

\begin{table}[H]
\centering
\caption{Produto dos polinômios de base do NEM.}
\label{chap3:ksub:table}
\vspace{0.5cm}
\begin{tabular}{|>{\centering} m{2cm}|>{\centering} m{2cm}|>{\centering} m{2cm}|>{\centering} m{2cm}|>{\centering\arraybackslash} m{2cm}|}
\hline
\diagbox[innerwidth=2cm]{$m$}{$k$}        & \textbf{1}    & \textbf{2}    & \textbf{3}     & \textbf{4}      \\ \hline 
\textbf{1} & $\frac{1}{3}$ & $0$           & $\frac{1}{5}$  & $0$             \\ \hline 
\textbf{2} & $0$           & $\frac{1}{5}$ & $0$            & $-\frac{3}{35}$ \\ \hline
\textbf{3} & $\frac{1}{5}$ & $0$           & $\frac{6}{35}$ & $0$             \\ \hline
\textbf{4} & $0$           & $0$           & $0$            & $\frac{6}{105}$ \\ \hline
\end{tabular}
\end{table}
\subsection{@neo4j/graphql}
A GraphQL to Cypher query execution layer for Neo4j and JavaScript GraphQL implementations.

\section{Apollo}

\section{Node.js}

\label{chap3:sec:fluxograma}

A Figura \ref{chap3:fluxograma} In ornare, enim non porta interdum, est lorem volutpat metus, pellentesque pharetra lacus est sed lacus. Vivamus quis magna et justo mattis commodo viverra in tellus. Cras tempor ullamcorper libero vitae tristique. Morbi malesuada posuere tincidunt. Integer accumsan egestas ante eget elementum. Vestibulum ante ipsum primis in faucibus orci luctus et ultrices posuere cubilia curae; Curabitur ac lacinia urna. Vivamus id nunc a nisl tincidunt efficitur eget quis neque. Praesent quis lorem rhoncus, rhoncus dui vel, condimentum dolor. Curabitur condimentum augue dignissim turpis consectetur venenatis.

\section{Vue.js}


\begin{figure}[H]

\begin{tikzpicture}[font=\small,thick]
 
% Start block
\node[draw,
    rounded rectangle,
    minimum width=2.5cm,
    minimum height=1cm] (block1) {Início};
 
% Voltage and Current Measurement
\node[draw,
    trapezium, 
    trapezium left angle = 65,
    trapezium right angle = 115,
    trapezium stretches,
    below= 0.8cm of block1,
    minimum width=3.5cm,
    minimum height=1cm
] (block2) { Geometria, $N_i^n{(t_1)}$, $K_{\text{sub}}^\text{Alvo}(t_1)$, $\Delta K_{\text{sub}}^\text{Alvo}$, $L$, $\varepsilon_0$ e $P_\text{Nominal}$ ou $\overline{I}_{s,0}$};
 
% Power and voltage variation
\node[draw,
    below=0.8cm of block2,
    minimum width=2.5cm,
    minimum height=1cm
] (block3) { $l=1$};

% Power and voltage variation
\node[draw,
    below=0.8cm of block3,
    minimum width=3.5cm,
    minimum height=1cm
] (block4) {Cálculo de $\hat{\phi}_g^n(t_l)$ e $\overline{\Psi}_g^{*n}(t_l)$ usando o NEM};

% Voltage and Current Measurement
\node[draw,
    trapezium, 
    trapezium left angle = 65,
    trapezium right angle = 115,
    trapezium stretches,
    left=of block4,
    minimum width=3.5cm,
    minimum height=1cm
] (block21) {Dados Nucleares};

% Power and voltage variation
\node[draw,
    below=0.8cm of block4,
    minimum width=3.5cm,
    minimum height=1cm
] (block5) {Cálculo de $K_{\text{sub}}$};
    
% Conditions test
\node[draw,
    diamond,
    below=0.8cm of block5,
    minimum width=2.5cm,
    inner sep=0] (block6) {$\left|\frac{K_{\text{sub}}(t_l)}{K_{\text{sub}}^\text{Alvo}(t_l)} -1 \right| \le \varepsilon_0$};
    
    
% Conditions test
\node[draw,
    diamond,
    below=of block6,
    minimum width=2.5cm,
    inner sep=0] (block7) {Calcula $P(t_l)$?};   

% Power and voltage variation
\node[draw,
    below left=0.5cm of block7,
    minimum width=3.5cm,
    minimum height=1cm,
    text width=4cm
] (block8) {Cálculo de $\overline{I}_s(t_l)$, se $P(t_l)=P_\text{Nominal}$ é dado};  

% Power and voltage variation
\node[draw,
    below right=0.5cm of block7,
    minimum width=3.5cm,
    minimum height=1cm,
    text width=3.5cm
] (block9) {Cálculo de $P(t_l)$, se $\overline{I}_s(t_l)=\overline{I}_{s,0}$ é dado};  

% Power and voltage variation
\node[draw,
    right= 1cm of block5,
    minimum width=3.5cm,
    minimum height=1cm,
    text width=3cm,
    text centered
] (block17) {Ajuste da Subcriticalidade};  

\node[coordinate,below=1.5cm of block7] (block16) {};

% Conditions test
\node[draw,
    diamond,
    below= of block16,
    minimum width=2.5cm,
    inner sep=0] (block10) {$l>L$};
  

%\node[coordinate,right=0.5cm of block17] (block18) {};
    

% Power and voltage variation
\node[draw,
    right=3.5cm of block10,
    minimum width=3.5cm,
    minimum height=1cm
] (block12) {$l=l+1$};

% Power and voltage variation
\node[draw,
    above=4cm of block12,
    minimum width=3.5cm,
    minimum height=1cm
] (block13) {$K_{\text{sub}}^\text{Alvo}(t_l) = K_{\text{sub}}^\text{Alvo}(t_1) + \frac{t_l}{t_L}\Delta K_\text{sub}^\text{Alvo}$}; 

% Power and voltage variation
\node[draw,
    above=0.8cm of block13,
    minimum width=3.5cm,
    minimum height=1cm
] (block14) {Calcula $\overline{\phi}_g^n(t_l) = \overline{I}_s(t_l)\hat{\phi}_g^n(t_l)$};    


% Power and voltage variation
\node[draw,
    above=0.8cm of block14,
    minimum width=3.5cm,
    minimum height=1cm
] (block18) {Calcula $N_i^n(t_l)$};    
    
% Return block
\node[draw,
    rounded rectangle,
    below=of block10,
    minimum width=2.5cm,
    minimum height=1cm,] (block11) {Fim};
 
% \node[coordinate,below=4.35cm of block4] (block12) {};
 
 
% Arrows
\draw[-latex] (block1) edge (block2)
    (block2) edge (block3)
    (block3) edge (block4)
    (block21) edge (block4)
    (block4) edge (block5)	
    (block5) edge (block6);
 
\draw[-latex] (block6) -- (block7)
    node[pos=0.5,fill=white,inner sep=0]{Sim};
    
\draw[-latex] (block6) -| (block17)
    node[pos=0.25,fill=white,inner sep=0]{Não};    
 
\draw[-latex] (block7) -| (block8)
    node[pos=0.25,fill=white,inner sep=0]{Sim};

\draw[-latex] (block7) -| (block9)
    node[pos=0.25,fill=white,inner sep=0]{Não};

\draw (block8) |- (block16);
 
\draw (block9) |- (block16);

\draw[-latex] (block16) -- (block10);

\draw[-latex] (block10) -- (block11)
    node[pos=0.5,fill=white,inner sep=0]{Sim};

\draw[-latex] (block10) -- (block12)
    node[pos=0.5,fill=white,inner sep=0]{Não};


\draw[-latex] (block12) edge (block13)
              (block13) edge (block14)
              (block18) |- (block4)
              (block17) |- (block4)
              (block14) edge (block18);
%\draw (block17) -- (block18);              
 
\end{tikzpicture}
\vspace{-1cm}
\caption{Fluxograma do algoritmo de solução para determinar o parâmetro $K_{\text{sub}}$.}
\label{chap3:fluxograma}
\end{figure}
