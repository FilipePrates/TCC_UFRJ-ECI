\chapter{Descrição do Sistema ``Admins´´}
\label{chap4}

Dado a escolha de um banco de dados em grafo, como o Neo4j, o Sistema Admins veio para resolver o problema de ser complicado e pouco seguro gerenciar os dados diretamente através de uma Cypher no banco, sendo facilmente possível realizar consultas muito pesadas, ou que tem consequências difíceis de reverter, de maneira acidental.

Foi então desenvolvido uma interface para usuários interagirem com o Banco, podendo criar, e conectar arbitrariamente nós no banco de dados. Esse sistema foi nomeado de 'admins', graças ao subdomínio utilizado para hospoedá-lo.

\section{Perfis e Listas}

Junto com o time de Design, identificamos uma maneira de entender e visualizar o grafo no banco, dando para cada nó uma página de perfil, que mostra e permite edição de suas propriedades e relações, além de uma página de lista/pesquisa para cada rótulo (relevante) no banco de dados. Endentendo este isomorfismo do grafo com esses perfis, conseguimos criar esta interface e permitir os usuários navegarem e editarem o grafo de maneira intuitiva.

percebemos que no fundo só precisamos de implementar uma só "pagina de perfil"


\section{Arquitetura da Aplicação Cliente}


\section{Arquitetura do Servidor com endpoint GraphQL}

O frontend, entretando, não se comunica diretamente com o banco de dados, a interface se comunica com uma camada intermediária através de requisições em GraphQL, e esta camada intermediária gera a Cypher resultante, que então é executada no banco de dados. Nesta seção descreveremos como funciona e a arquitetura escolhida para esta camada intermediária, o servidor que disponibiliza a endpoint em Graphql.
 
MC Poze é conhecido pelas letras polêmicas, características do subtipo conhecido como ``funk proibidão''. O envolvimento em processos relacionados a sua participação com organizações ligadas ao tráfico de drogas é situação presente na história de vida do cantor. Frequentemente, o fato de suas letras conterem aparente exaltações ao modus operandi de facções criminosas é apontado como problemático.A réplica por parte do artista geralmente orbita na questão do mesmo se colocar apenas como um veículo que retrata a realidade das comunidades carentes do Rio de Janeiro por meio da música.